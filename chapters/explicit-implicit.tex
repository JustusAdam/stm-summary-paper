\section{Manual and Automatic STM}

\label{sec:explicit-implicit}

The problem with manual STM, where the programmer inserts \texttt{READ} and
\texttt{WRITE} calls, is that these can be forgotten without raising an error.
Special STM compilers are employed to automate the insertion of \texttt{READ}s
and \texttt{WRITE}s this is often also necessary to instrument third party
libraries that are used inside transactions. Since the STM protected accesses
are much more expensive than regular memory accesses protecting data that is not
shared should be avoided. The process of data that was shared becoming local to
the thread is called \emph{privatization}. Identifying which is private is not
trivial and involves sophisticated escape analysis. As a result STM compilers
are generally too conservative and protect additional memory that is not shared.
This further reduces their performance.

Manual STM can be used to avoid this excess protection. However in such cases it
is easy to forget protecting actually shared memory, because most languages make
no type level distinction between protected and unprotected memory. In Haskell
STM protected memory structures are distinct types. The simplest is a mutable
memory cell called \texttt{TVar} (see Figure~\ref{fig:tvar}). Its internal
structure is hidden from the user of the library which prevents any raw
manipulation of a \texttt{TVar} value. It can only be manipulated via the
functions provided by the library. These \texttt{readTVar} and \texttt{putTVar}
functions internally use \texttt{READ} and \texttt{WRITE} to protect the memory.

No special STM compiler is needed here. The programmer uses STM reads and writes
explicitly and the restricted library interface ensures that. It also means safe
privatization. While the \texttt{TVar} is itself mutable the value stored there
is either immutable or another explicit \texttt{TVar}. Once the value of a
\texttt{TVar} has been read its members can be accessed without using the costly
STM protection. If one of the members happens to be mutable (aka a
\texttt{TVar}) only \emph{that} access is again protected by the system.

The type system further ensures \texttt{TVar}s are only used inside
transactions. Modification of shared, protected memory from outside a
transaction is a problem for stm implementations~\cite{research-toy}. Most
implementations simply do not protect against this. As shown in
Figure~\ref{fig:tvar} the functions for manipulating \texttt{TVar}s all have the
\texttt{STM} type which can only be used inside a transaction.

\texttt{TVar}s themselves can be returned from a transaction. For instance by
doing \texttt{atomically (newTVar 0) :: IO (TVar Int)}. This way \texttt{TVar}s
can be passed around outside of transactions, but only read or written to in
another transaction.

Having only manual STM in Haskell unfortunately means that STM cannot easily
applied to third party libraries. However this is rarely a problem in practice,
because many libraries in Haskell tend to be designed in terms of pure functions
which means they do not need to involve STM.
