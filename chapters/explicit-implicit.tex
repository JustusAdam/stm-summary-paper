\section{Explicit and Implicit STM}

\label{sec:explicit-implicit}

The inner workings can be quite different between STM implementations. Some use
a logging approach, where changes are recorded but not executed. An example of
this would be the \texttt{stm} Haskell library. Others use a blocking and help
approach. A thread $T_1$ that wants to write to a location $l$ that another
thread $T_2$ has already modified first helps $T_2$ until $T_2$ commits and then
continues. In either case runtime bookkeeping is necessary to record the
locations read and the modifications made.

The following four actions are the basic stm interface.

\begin{description}
\item[START\_TRANSACTION] Initialises the bookkeeping records for the
  transaction. Return point in case of an abort.
\item[READ] Read from an STM protected memory location.
\item[WRITE] Write to an STM protected memory location.
\item[END\_TRANSACTION] Verifies the transaction and either commits or retries.
\end{description}

It is important that all reads and writes to shared memory are done using the
\texttt{READ} and \texttt{WRITE} actions. In languages which support raw memory
manipulations a programmer might forget to use them or be unaware that a certain
memory location is shared. This causes the transaction to leak, which might put
the system in an inconsistent state when an abort occurs.

To deal with this problem special stm compilers have been designed, which
automatically insert \texttt{READ} and \texttt{WRITE} calls in transactions.
They cannot however ensure that accessing those locations outside of
transactions is done safely. In addition these compilers are often too
conservative and also protect memory that is not shared. STM reads and writes
are always costly, regardless of the implementation. Protecting this additional
memory makes the program slower.



The STM protected, mutable memory cell is called \texttt{TVar} (see
Figure~\ref{fig:tvar}). The \texttt{readTVar} and \texttt{putTVar} functions
internally use \texttt{READ} and \texttt{WRITE} to protect the memory. Crucially
the implementation of \texttt{TVar} is hidden from the programmer. The only way
of interacting with the data is through the defined interface which uses
\texttt{READ} and \texttt{WRITE}. This way Haskell can use the more efficient
explicitly annotated STM, without the possibility of accidental unprotected
access.

In addition to that, the type system also ensures \texttt{TVar}s are only used
inside transactions. Modification of shared, protected memory from outside a
transaction is a problem for stm implementations~\cite{research-toy}. Most
implementations simply do not protect against this. As shown in
Figure~\ref{fig:tvar} the functions for manipulating \texttt{TVar}s all have the
\texttt{STM} type. \texttt{STM} is the type of STM transactions. \texttt{STM} is
an effect type like \texttt{IO}. It can also be composed into larger
transactions using \texttt{do}. The \texttt{atomically :: STM a -> IO a}
function is used to execute the transaction, and make its final result available
to the outside. A sketch of how \texttt{atomically} works can be seen in
Figure~\ref{fig:atomically}.

\begin{figure}
  \begin{lstlisting}
atomically :: STM a -> IO a
atomically transaction = do
  START_TRANSACTION
  result <- runTransaction transaction
  END_TRANSACTION
  return result
  \end{lstlisting}
  \caption{A rough version of the \texttt{atomically} function}
  \label{fig:atomically}
\end{figure}

\texttt{TVar}s themselves can be returned from a transaction. For instance by
doing \texttt{atomically (newTVar 0) :: IO (TVar Int)}. This way \texttt{TVar}s
can be passed around outside of transactions, but only read or written to in
another transaction.
