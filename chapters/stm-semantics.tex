\section{Semantics of Transactional Memory}

\label{sec:stm-semantics}

Within a transaction every read to a shared memory location is recorded and
writes are performed tentatively, for instance to a local cache, until the
transaction commits. Upon commit all writes become visible atomically to other
threads. During the commit phase a thread ensures that the reads it performed
before are still valid, meaning it saw a consistent, unmodified view of the
memory. If a concurrent transaction committed in the meantime and modified a
memory location read by the thread, the transaction is aborted and started anew.
Conversely a transaction will only ever abort if a memory modification has taken
place. Memory modifications can only ever take place if a transaction has
committed, meaning that even though one transaction may have to be aborted, the
system as a whole has still made progress because some other transaction
finished.

This guarantees deadlock freedom and liveness for transactional systems.
Starvation however may still occur, if a short running transaction repeatedly
forces longer running transactions to abort.

Side effects are a particularly difficult problem in memory transactions. Since
a transaction may be aborted multiple times and re-run. Any side effects such as
I/O will be executed again and possibly lead to different results. Similarly
writes to shared memory locations not protected by the transactional memory
system may lead to information leakage out of the transaction and inconsistent
system states.

Therefore both manipulation of non protected shared memory and I/O should not be
used in transactions, though very few systems enforce this. An exception being
the Haskell STM library, which disallows non-stm side effects via the type system.
