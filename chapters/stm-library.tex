\section{The Haskell STM Library}

\label{sec:library}

In Haskell STM is realized via a library called \texttt{stm}. Whereas the API is
defined in the library the core of Haskells STM system is implemented in its
runtime, in C.

The library is designed around explicit transactional data structures. The
simplest of these is called \texttt{TVar}. The interface for interacting with a
tvar can be seen in Figure~\ref{fig:tvar}. It is remarkably similar to the
\texttt{IORef} from the previous section. The entire library mimics preexisting
Haskell primitives, such as the transactional lock \texttt{TMVar}, which is
similar to an \texttt{MVar}. Because the interface is so similar every Haskell
programmer should have little difficulty using it.

The second important part of the library is the transaction type \texttt{STM}.
\texttt{STM} is an effect type, similar to \texttt{IO}. It is used to build up
transactions from primitive operations using the \texttt{do} notation. It
combines both primitives, as well as other fragments of transactions together.
Thus libraries can define only partial transactions which the user can combine
later with their own logic using \texttt{do}.

To execute the transaction the \texttt{atomically} function is provided, see
also Figure~\ref{fig:atomically}. It takes a description of a transaction
(\texttt{STM a}) and executes it. The \texttt{atomically} function both
initializes the transaction as well as verifies it. See more in
Section\ref{sec:explicit-implicit}.

\begin{figure}
  \begin{lstlisting}
data TVar = {- implementation hidden -}
newTVar :: a -> STM (TVar a)
readTVar :: TVar a -> STM a
putTVar :: TVar a -> a -> STM ()
  \end{lstlisting}
  \caption{STM memory cell in Haskell}
  \label{fig:tvar}
\end{figure}

The Haskell \texttt{stm} library exports two additional primitives.
\texttt{retry} and \texttt{orElse}. These are typically not part of STM and were
first introduced in \cite{composable-transactions}. \texttt{retry} is a way for
the user to demand the transaction be aborted and attempted again. This function
allows the programmer to require more complex conditions to hold in the system.
Take for instance a function which must read two elements from a shared buffer.
The implementation in Figure~\ref{fig:dequeue-two} uses retry if any of the
buffers are empty. When a \texttt{retry} is called the STM implementation rolls
back the transaction and then waits for a change to occur on any of the
locations touched until \texttt{retry} as called. This avoids executing the
computation again when the conditions have not changed.

\begin{figure}
  \begin{lstlisting}
dequeue :: TMVar [a] -> STM a
dequeue v = do
  l <- takeTMVar v
  case l of
    [] -> retry
    (a:as) -> do
      putTMVar v as
      return a

dequeueTwo :: TMVar [a] -> STM (a,a)
dequeueTwo v = do
  a1 <- dequeue v
  a2 <- dequeue v
  return (a1, a2)
  \end{lstlisting}
  \caption{Dequeuing a buffer}
  \label{fig:dequeue-two}
\end{figure}

The second primitive added is the \texttt{orElse} function. \texttt{orElse}
takes two STM computation, tries the first and if it \texttt{retry}s it
executes the second. If we understand \texttt{retry} as a blocking transaction,
because it waits for change, then \texttt{orElse} is the choice of two blocking
transactions. By using \texttt{orElse} we can wait on two different conditions
for either one to come true. The authors of the original paper note the
similarity to the unix \texttt{select} system call.