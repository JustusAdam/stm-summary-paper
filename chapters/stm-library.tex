\section{The Haskell STM Library}

\label{sec:library}

Haskell exposes its STM implementation via a library called
\texttt{stm}~\cite{composable-transactions}. The underlying STM system is
implemented in the Haskell runtime and similar to the one described in
Section~\ref{sec:stm}. Though it has been shown that a it can also be
implemented in Haskell itself, using lower level concurrency
primitives~\cite{stm-in-concurrent-haskell,concurrent-haskell}.

The library exposes a high level interface to the
programmer. The advantage is that it can use the Haskell type system to enforce
correct use of transactional data and side effect freedom.
Sections~\ref{sec:side-effects}~and~\ref{sec:explicit-implicit} describe these
in more detail.

The library is designed around explicit transactional data structures. The
simplest of these is called \texttt{TVar}, a mutable memory cell that can be
read or written. The interface for interacting with a \texttt{TVar} can be seen
in Figure~\ref{fig:tvar}. The interface for \texttt{TVar} is deliberately
similar to the one for \texttt{IORef} from Section~\ref{sec:haskell}. The entire
library mimics preexisting, non-transactional Haskell data structures. The
transactional lock \texttt{TMVar} for instance is similar to an \texttt{MVar}, a
non-transactional lock. By making names and interfaces similar to
familiar data structures \texttt{stm} reduces the barrier to entry for Haskell
programmers already familiar with concurrent Haskell~\cite{concurrent-haskell}.

The second important part of the library is the transaction type \texttt{STM}.
\texttt{STM} is an effect type, similar to \texttt{IO}. It is used to build up
transactions from primitive operations using the \texttt{do} notation. It
combines both primitives, as well as other fragments of transactions together.
Thus libraries can define only partial transactions which the user can combine
later with their own logic using \texttt{do}.

\begin{figure}
  \begin{lstlisting}
atomically :: STM a -> IO a
atomically transaction = do {
  START_TRANSACTION;
  result <- runTransaction transaction;
  END_TRANSACTION;
  return result
}
  \end{lstlisting}
  \caption{A rough version of the \texttt{atomically} function}
  \label{fig:atomically}
\end{figure}

To execute the transaction the \texttt{atomically} function is provided, see
also Figure~\ref{fig:atomically}. It takes a description of a transaction
(\texttt{STM a}) and executes it. The \texttt{atomically} function both
initializes the transaction as well as verifies it.

\begin{figure}
  \begin{lstlisting}
data TVar = {- implementation hidden -}
newTVar :: a -> STM (TVar a)
readTVar :: TVar a -> STM a
putTVar :: TVar a -> a -> STM ()
  \end{lstlisting}
  \caption{STM memory cell in Haskell}
  \label{fig:tvar}
\end{figure}

The Haskell \texttt{stm} library exports three additional primitives.
\texttt{retry}, \texttt{orElse} and \texttt{check}. These were first to appear
in this
library~\cite{composable-transactions,transactional-memory-data-invariants} and
are not part of the standard STM interface. \texttt{retry} is a way for the user
to demand the transaction be aborted and attempted again. This function allows
the programmer to require more complex conditions to hold in the system. Take
for instance a function which must read two elements from a shared buffer. The
implementation in Figure~\ref{fig:dequeue-two} uses \texttt{retry} if any of the
buffers are empty. When a \texttt{retry} is called the STM implementation rolls
back the transaction and then waits for a change to occur on any of the
locations touched until \texttt{retry} is called. This avoids executing the
computation again when the conditions have not changed. The use of a
\texttt{TMVar} as a transactional lock here is safe, as it is tracked by the STM
runtime and automatically released should the transaction abort.

\begin{figure}
  \begin{lstlisting}
dequeue :: TMVar [a] -> STM a
dequeue v = do {
  l <- takeTMVar v; -- acquire lock
  if empty l
    then retry
    else do {
      putTMVar v (tail l); -- release lock
      return (head l)
    }
}

dequeueTwo :: TMVar [a] -> STM (a,a)
dequeueTwo v = do {
  a1 <- dequeue v;
  a2 <- dequeue v;
  return (a1, a2)
}
  \end{lstlisting}
  \caption{Dequeuing a buffer}
  \label{fig:dequeue-two}
\end{figure}

The second primitive added is the \texttt{orElse} function. \texttt{orElse}
takes two STM computation, tries the first and if it \texttt{retry}s it
executes the second. If we understand \texttt{retry} as a blocking transaction,
because it waits for change, then \texttt{orElse} is the choice of two blocking
transactions. By using \texttt{orElse} we can wait on two different conditions
for either one to come true. The authors of the original paper note the
similarity to the unix \texttt{select} system call.

Lastly the library introduces the \texttt{check} function which allows the
programmer to specify an ``expression that should be preserved by every atomic
update for the remainder of the program’s
execution''~\cite[abstract]{transactional-memory-data-invariants}. This allows the
user to programmatically state conditions which must be preserved in the
transaction. The STM system can then optimise waiting and execution of
transactions to suite these conditions.

To sum up, Haskell exposes its STM implementation via a library. The interface
is high level and mimics similar, non-transactional data structures. In addition
to the ``regular'' STM primitives it introduces thee new functions which add
blocking, choice and programmatic condition checking to transactions. Neither of
these three is part of the standard STM interface. They enable users to express
even more complex computations in a succinct way and delegate the responsibility
for efficiency to the STM system. Concrete examples can be found in the two
publications~\cite{composable-transactions,transactional-memory-data-invariants}.
