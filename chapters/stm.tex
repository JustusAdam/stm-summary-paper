\section{Software Transactional Memory}

\label{sec:stm}

Transactional memory is a tool to do concurrent programming without locks. Locks
are notoriously easy to program incorrectly. They also do not compose well.
When multiple lock protected data structures are used simultaneously, the order
in which the their locks are acquired is important. When two sections of code
acquire the same locks in a different order it can easily lead to a deadlock.

Transactional memory uses no locks. Instead it expands the idea of atomic
instructions to whole sections of code. Sequences of memory reads and writes are
grouped into \emph{transactions}. When a transaction runs, none of its writes
can be seen by the other threads. The transaction finishes by attempting a
\emph{commit}. This commit either \emph{succeeds}, making \emph{all} of the
transactions writes visible to the other threads atomically, or \emph{aborts},
discarding all writes and restarting. Thus to all other threads it seem like the
transaction either happened instantly or it never took place at all. Whether a
transaction succeeds or aborts depends on whether the memory it has \emph{read}
while running has changed in the meantime. In this case the transaction may have
read an inconsistent view of the heap. It therefore must be considered unsafe
and restarted.

Importantly transactional memory never stalls a system, like a deadlock would.
So long as there is no conflict each transaction can proceed freely. Upon
detecting a conflict is must abort. However conflicts only arise when memory
changes\footnote{This is not strictly true for all implementations. Some choose
  to detect conflicts with other concurrent, non-committed transactions. Then
  the aborting thread starts helping the other transaction. While it is a
  subtile difference, the outcome is the same. Aborts only ever happen to the
  benefit of another transaction and thus the whole system makes progress.}, and
memory only changes when transactions commit. Therefore in case of an abort,
another transaction must have committed in the meantime, which means the system
as a whole \emph{must} have made progress.

There are two major types of transactional memory.
\begin{description}
\item[HTM] \textbf{H}ardware \textbf{T}ransactional \textbf{M}emory uses
  hardware to implement transactions.
\item[STM] \textbf{S}oftware \textbf{T}ransactional \textbf{M}emory implements
  transactions in software.
\end{description}

HTM is the older system, first introduced by
\citeauthor{tm-origins}~\cite{tm-origins}. Its major advantage is that it is
fast, because hardware is fast. STM is a little younger, developed by
\citeauthor{stm}~\cite{stm}. Its major advantage over HTM is that it is
portable. Not only do some systems lack the support for HTM entirely but there
can be subtile differences between different hardware. Transactional memory
always involves some kind of bookkeeping. Because HTM uses hardware for this
bookkeeping, its maximum transaction \emph{size} (or \emph{length}) is limited
by hardware capacity. A program using HTM that runs fine on one system may
suddenly error on another due to a transaction that is too large. STM does not
have this limitation and can be ported to other hardware without such subtile
bugs but has larger overhead in general, making it slower.

The inner workings can be quite different between transactional memory
implementations. For the sake of brevity I will not go too deep into these differences,
but instead focus on the commonalities that are of interest to this paper.

Independent of the concrete algorithm used, transactional memory involves
bookkeeping that keeps track of both reads and writes
\begin{description}
\item[Reads] must be kept track of because at some point each transaction must verify
that it has seen a consistent view of memory. To do this the locations read must
be checked for intermediate modification.
\item[Writes] must be kept track of to allow transactions to be undone.
  Optimistic strategies write directly and undo them with an undo log in case of
  an abort. Pessimistic strategies record writes in a write log that is
  persisted after a successful commit.
\end{description}

To facilitate this the following basic STM user interface is needed.
\begin{description}
\item[START\_TRANSACTION] Initialises the bookkeeping records for the
  transaction. Return point in case of a retry.
\item[READ] Read from an STM protected memory location.
\item[WRITE] Write to an STM protected memory location.
\item[END\_TRANSACTION] Verifies the transaction and either commits or retries.
\end{description}

It is crucial that all reads and writes to shared memory are done using
\texttt{READ} and \texttt{WRITE}. In languages which support raw memory
manipulations a programmer might forget to use them or be unaware that a certain
memory location is shared. This causes the transaction to leak, which might put
the system in an inconsistent state especially when parts of the transactions
are executed repeatedly due to aborts.

To deal with this problem special STM compilers have been designed, which
automatically insert \texttt{READ} and \texttt{WRITE} calls in transactions.

There is an ongoing debate~\cite{research-toy,not-research-toy} about how
feasible STM is for real world applications. The argument can be summed up in
the following major points.

\begin{description}
\item[Efficiency] STM in particularly suffers from the additional cost of
  bookkeeping. This has improved though, due to smarter STM compilers employing
  escape analysis to recognize private data. By privatizing local data
  structures some of the overhead can be eliminated.
\item[Weak atomicity] STM compilers can ensure that accessing shared memory in
  transactions is done safely, however it cannot guarantee that the shared
  memory locations are not accessed unprotected from outside the transaction.
\end{description}

We will revisit these problems starting with Section~\ref{sec:library} to
see how the Haskell STM library addresses these.