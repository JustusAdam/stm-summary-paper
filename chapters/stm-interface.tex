\section{STM interfaces}

\label{sec:stm-interfaces}

In addition to the calls to start and commit an STM transaction, using the
transaction also requires changes to be made to the memory accesses themselves.
Depending on the implementation different addition steps have to be executed
when accessing shared memory. As an example in the general STM described in
\cite{stm-research-toy} reads incur an additional metadata write to the read
set, as well as a validation.

Generating the necessary code can either be done automatically by a special stm
compiler or, alternatively, manually Using the compiler based solution often
results in a more conservative approach with redundant barriers for memory
locations only available to the local thread. The manual approach can be
optimised and have less overhead but carries a similar risk as lock based
approaches, where a skilled programmer is required with intimate knowledge of
the system, the data sources and their interactions. One potential pitfall for
instance is that not only data being \emph{written} by the transaction have to
be protected, but also data which is only \emph{read} but could be modified by a
concurrent thread. This becomes harder to track as the system gets larger.

One elegant way to avoid this problem is the approach taken by the Haskell STM
library, where transactional variables have to explicitly created and can only
be accessed inside transactions. A special \texttt{STM} monad is used to
describe actions which have to take place in transactions. Such actions can only
be run using the \texttt{atomically} function which wraps the action in a
\texttt{START_TRANSACTION} and \texttt{END_TRANSACTION} call.
