\section{Haskell}

\label{sec:haskell}

Many of the examples throughout this paper use Haskell syntax. Most prominently
for type annotations. Following therefore is a short introduction to selected
aspects of the Haskell syntax.

Values are annotated with types using the \texttt{value :: Type} syntax,
regardless of whether it is a function or other value. Function types are
expressed with the \texttt{->} which separates both the arguments and the return
type. The type to the right of the last \texttt{->} is always the return type,
all other types are arguments. Thus \texttt{f :: Int -> String -> Bool} reads as
\texttt{bool f(int i, string s)} in C/C++. Perhaps confusing is the omission of
parameter names in the Haskell type signature. Applying functions to arguments
in Haskell is done in simple prefix notation, with no parentheses and no commas.
Thus \texttt{writeFile("filename", content)} would translate to
\texttt{writeFile "filename" content}.

In the Haskell data is immutable by default. Regular data structures cannot be
modified, instead new data is allocated. Similarly binding ``variables'' is
final. The \texttt{let x = 1} syntax declares a new variable \texttt{x}. Though
``variable'' may be the wrong word here, because there is no way to change this
value. Instead new data must be created, for instance \texttt{let y = x + 1}.

In addition side effects are strictly controlled. Functions are either pure or
must explicitly declare the side effects they perform. Pure functions have the
property that for the same inputs they always produce the same outputs. They
cannot modify any external state, perform I/O or read mutable variables. As an
example a function \texttt{randomInt :: () -> Int} producing a random integer
would be impossible in Haskell.

The return type of a function shows the side effects it performs. For instance
\texttt{readFile :: FilePath -> IO String} does I/O side effect by opening and
reading a file, and then returns a \texttt{String}. There are also functions
which only perform side effects. They return \texttt{()}, the Haskell equivalent
of \texttt{void}. An example would be \texttt{writeFile :: FilePath -> String ->
  IO ()}. The \texttt{IO} ``wrapper'' type around the string cannot be removed.
However Haskell provides the \texttt{do} notation to sequence many such
computations into larger ones. Inside the \texttt{do} block values produced by
I/O can be captured and assigned to names with \texttt{<-}. These values can be
manipulated as usual. However the \texttt{IO} type propagates to the return type
of the computation. As an example see Figure~\ref{fig:simple-do}.


\begin{figure}
  \begin{lstlisting}
cCompile :: IO ()
cCompile = do {
  header <- readFile "main.h";
  code <- readFile "main.c";
  let object = runCompiler header code;
  writeFile "main.o" object;
  return ()
}
  \end{lstlisting}
  \caption{Example for the \texttt{do} notation}
  \label{fig:simple-do}
\end{figure}


The only way in Haskell to perform the actual effects of an \texttt{IO}
computation is to use it in the \texttt{main} function. Because effect types,
such as \texttt{IO}, propagate automatically, we know that a pure function can
never perform I/O.

In Haskell raw memory manipulations are not supported. Instead there are special
data structures which represent mutable memory cells, called \texttt{IORef}.
Instead of primitive, syntactic constructs, there is a suite of functions to
manipulate them, shown in Figure~\ref{fig:io-ref}.

\begin{figure}
  \begin{lstlisting}
data IORef = {- implementation hidden -}
newIORef :: a -> IO (IORef a)
readIORef :: IORef a -> IO a
putIORef :: IORef a -> a -> IO ()
  \end{lstlisting}
  \caption{Mutable memory cell in Haskell}
  \label{fig:io-ref}
\end{figure}

These manipulations are also considered side effects, as can be seen from the
\texttt{IO} return type. As a result they also can only be used in an
\texttt{IO} context.
