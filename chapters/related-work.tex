\section{Related Work}

\label{sec:related-work}

While STM has not caught on much outside of Haskell, other concurrency
primitives have.

\subsection{HLE}

Hardware lock elision~\footnote{In the first occurrence of this idea it was
  called Speculative Lock Elision~\cite{sle}}, or HLE is a hardware mechanism
that speeds up lock based concurrent programs~\cite{sle,hle}. The idea is rather
similar to transactional memory. A lock based implementation is executed
optimistically without acquiring the lock. When the lock would be released a
check for conflicts is performed using a cache coherence protocol. ``Two memory
operations conflict if they concurrently access the same cache line and one of
them is a write.''~\cite{hle} If no conflict occurs the writes are persisted. If
a conflict occurs the threads re-execute the section non-speculatively. This
only works if there are no side effects such as IO in the lock protected
sections.

The problems are similar to TM, in that it needs hardware capabilities, and that
it is limited by the size of the cache. However unlike HTM, HLE is a runtime
feature of the processor. No changes to the code need to be performed, making it
suitable for use with legacy systems and third party libraries as well. This
also makes it, if not portable, compatible. On systems which do not support HLE
it automatically falls back to the less efficient, lock based solution.

\subsection{Message passing}

Message passing, and by extension the \emph{actor model} is most up-and-coming
concurrency model. This is often coupled with lightweight threads that are
scheduled by a runtime. Threads only interact with private state. Between them
there are channels through which data can be sent. While the paradigm itself is
quire old~\cite{mpi-paradigm,actors} it has only recently found its way into the
mainstream of programming. The first language to use this model extensively was
Erlang, a language specifically designed for very reliable, highly distributed
systems. It is not a widely used language, but neither irrelevant due to its use
in industry. Nowadays the standard library for Go and Rust both contain an
implementation of message passing. And it has also found its way onto the Java
platform, notably with the Akka framework~\cite{akka}.

The advantage of MP is that these principles are easy to understand and they
scale well. Some implementations, such as Erlang\cite{erlang}, require messages
to be serializable, an thus individual nodes can be moved to other processes or
even machines making them scalable beyond threads.

A common problem of MP is the serialization overhead of messages. This can be
avoided by sending pointers directly when communicating between threads. However
care must be taken that the access of the sending thread is revoked~\footnote{Go
  for instance allows you to send pointers through channels. This is efficient,
  but the recipient can modify the memory pointed to which, since it was sent
  may be shared. There is no mechanism in the compiler to detect and/or report
  this.}. Rust has a rather elegant solution here where its ownership type
system ensures that no two threads access the same data concurrently.

\subsection{Hybrid TM}

While not strictly a ``different'' concurrency primitive it is worth mentioning
that there are designs for so called hybrid transactional systems. These combine
HTM and STM. One way of doing this is by using HTM wherever possible and if the
the hardware capacity is exceeded fall back to a software based
solution~\cite{htm-with-stm-fallback}.

Alternatively hardware is used to speed up an STM
system~\cite{hardware-accelerated-stm}. Here a classical STM system is employed
but the instruction set of the machine is extended with new instructions
specifically designed to aid the STM implementation. The claim is that this way
near HTM performance can be achieved while maintaining the flexibility of STM.

None of these are seriously in use yet.