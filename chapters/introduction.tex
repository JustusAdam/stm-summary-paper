\section{Introduction}

\label{sec:introduction}

% Transactional memory is a much debated solution to dealing with the complexity
% of parallel and concurrent programs. It offers perhaps the most convenient
% interface for the programmer where code may be written in a style with hardly
% any deviation. Unlike lock based approaches a transactional program cannot
% deadlock and depending on the implementation also offers liveness guarantees.
% However this comes at a cost. To enable transactions an underlying system is
% needed which takes care of the bookkeeping necessary to perform transactions
% atomically.

Implementing concurrent programs correctly is a notoriously difficult task.
Simple lock based techniques easily result in deadlocks, performance loss due to
overly conservative locking, or inadvertent concurrent modification if a lock
was not acquired. Often the complexity of using locks cannot be hidden away when
more than one shared object is to be used together. Thus every use site of the
concurrent data structures has the potential for the above mentioned errors,
requiring intimate knowledge of the precise locking semantics of each structure
in order to be used correctly. It is fair to say that, in general, lock based
data structures and systems do not compose well and are not modularisable.

To alleviate this burden from the programmer a system based on the transactional
semantics from databases was proposed by \textbf{Cite original TM
  paper}~\cite{tm}.
% Perhaps go into more detail here
Leveraging hardware facilities to group memory accesses into
\emph{transactions}. Each transaction proceeds invisible to any other concurrent
thread. Once it finishes it is verified that the thread saw a consistent view of
the memory, meaning no other thread modified it in the meantime, and then all
writes are applied atomically. If the verification fails the entire transaction
is aborted and started anew, all modifications are discarded.

Transactional systems are immune to many of the bugs so prevalent in lock based
systems. Deadlocks are ruled out and the system is guaranteed to always make
progress. Section~\ref{sec:stm-semantics} details more of the semantics of
transactions and how they prevent deadlocks. Composing concurrent structures is
easier as well by using them in the same transaction.

The hardware based approach by \emph{insert authors} however falls short in two
respects \begin{enumerate*}\item the system is not portable and \item
  transaction size is bounded by cache size
% TODO make sure this is actually correct
\end{enumerate*}. To address this Shavit and Touitou developed the
first version of a transactional system implemented solely in
software~\cite{stm}. Since then various improvements to the system have been
developed, improving safety, efficiency and generality.
