\section{Side Effects}

\label{sec:side-effects}

In order for an STM system to function properly transactions must be free of
side effects. Side effects are actions that manipulate external state. This
includes reading and writing files but also manipulations of global memory
locations. The problem is that the semantics of side effects in transactions are
unspecified and it is easy to use them unsafely. When a transaction is retried
its side effects are executed again. Furthermore there is no guarantee
\emph{which} of its side effects are executed again. The transaction may only
run partially and then abort, executing only a part of the side effects. They
also interleave with the side effects from concurrent transactions. As because
of this programmers are generally discouraged from using side effects in
transactions.

Detecting the use of side effects is difficult. It is possible to detect the use
of syscalls at runtime. Systems which do so will usually abort with an error to
inform the user. Much more complicated is the detection of memory modifications.
For efficiencies sake it is necessary to allow raw memory modifications in
transactions. However writes must be restricted to memory created in the
transaction. Furthermore this memory must not have been written out to a
protected location. The usual tactic of STM compilers is to use escape analysis,
which is not a trivial problem.

Haskell deals with this problem differently. The type system and the standard
library in Haskell make side effects explicit (see Section~\ref{sec:haskell}).
Memory manipulations (see Figure~\ref{fig:io-ref}) and OS interactions (see
Section~\ref{sec:haskell}) have the \texttt{IO} type. This type is incompatible
with the type of memory transactions \texttt{STM} (see Figure~\ref{fig:tvar}).
Both of these types are effect types. By default effect types cannot be
interleaved or embedded. An explicit function must be provided to embed one
effect in another. \texttt{STM} effect can be embedded into \texttt{IO} effects
using \texttt{atomically} (compare Figure~\ref{fig:atomically}). However there
is no such function for the other direction. \texttt{embedIO :: IO a -> STM a}
would be an example for such a function. However authors of the \texttt{stm}
library deliberately do not include such a function. Furthermore since both the
implementation of \texttt{IO} as well as \texttt{STM} are hidden, the user
cannot simply define such a function themselves.

As a result the compiler will detect and report the use of \texttt{IO}
functions, and therefore the use of side effects, in an stm transaction. This
prevents the accidental use of side effect performing functions. Haskell is in
the fortunate position to already have explicit side effects which makes this
approach possible int he first place.
