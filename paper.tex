\documentclass[11pt,a4paper,sigconf]{acmart}

\usepackage{booktabs} % For formal tables
\usepackage[utf8]{inputenc}
\usepackage{listings}
\usepackage[inline]{enumitem}
\usepackage{FiraMono}

% Copyright
%\setcopyright{none}
%\setcopyright{acmcopyright}
%\setcopyright{acmlicensed}
\setcopyright{rightsretained}
%\setcopyright{usgov}
%\setcopyright{usgovmixed}
%\setcopyright{cagov}
%\setcopyright{cagovmixed}

\lstset{%
  language=Haskell,
  basicstyle=\small\ttfamily,
%  literate=
%  {->}{{$\rightarrow$}}1 {=>}{{$\Rightarrow$}}1 {>>=}{{$\bindOp$}}2
%  {>>}{{$\bindIgnOp$}}1 {<-}{{$\leftarrow$}}1
}

\begin{document}
\title{Finding a home for Software Transactional Memory}
% \subtitle{Using Freedom to Great Effect}

\author{Justus Adam}
\affiliation{Technische Universiät Dresden}
\email{justus.adam@tu-dresden.de}


\begin{abstract}
  Software Transactional Memory is widely known to be a safe and easy but also
  slow abtraction for writing concurrent programs. Some go so far as to call
  it a ``research toy''\cite{research-toy}.

  In truth STM really has not seen much adoption by the mainstream, with one
  exception. Haskell, the most popular functional programming language, supports
  STM and here it has seen widespread use by major projects and frameworks.

  In this work I explore why STM integrates well into the Haskell ecosystem and
  how some of the common issues when using STM are remedied by Haskells powerful
  type system and design patterns.
\end{abstract}

\maketitle

\section{Introduction}

\label{sec:introduction}

Writing concurrent, and by extension parallel, programs is anecdotally hard.
Atomic operations make it possible to modify small sections of memory safely.
However they are insufficient when invariants between large or separated memory
sections must be maintained.

A simple example would be a vector consisting of a buffer ($b$) of memory to
store its element and a field ($l$) for the lengthof the vector. When a new
element is appended it has to be written into the buffer $b[l + 1]$, then $l$
has to be incremented. The two locations, $b$ and $l$ are separate memory
locations with the invariant that $l$ always corresponds to the smallest index
where $b$ no longer contains readable data. This invariant has to be maintained
in a concurrent setting, where the following scenario is possible. Two
concurrent threads attempt to perform an append. Both threads read the same
value for $l$. Both threads write their data to $b[l+1]$, whereby the second
thread overwrites the value written by the first. Lastly they either both set $l
= l + 1$, or, worse, execute an atomic increment each, resulting in $l = l + 2$,
with $l+1$ now being an invalid index containing garbage data.

A tool commonly used to solve this issue are \emph{locks}. When used correctly
they ensure a certain section of code is never executed by more than one thread
at a time. However locks are easily used incorrectly. Most systems do not tie
the use of the data to the lock that protects it. This means they cannot detect
when the programmer forgets to lock before accessing the data or unlock after it
goes out of scope. Also when more than one lock is used the system can
\emph{deadlock}, a state, where two threads hold a lock the respective other
thread needs also. In such a case the whole system can halt without any explicit
error.

\citeauthor{tm-origins} proposed a different system~\cite{tm-origins}, based on
the concept of \emph{transactions}. Transactions are widely used in databases to
solve a similar problem. It maintains invariants between tables in a database by
executing a sequence of instructions as if it were a single one. This prevents
the interleaving of concurrent modifications. \citeauthor{tm-origins} derived from
this a system for using memory in a transactional way. More detail about this
follows in Section~\ref{sec:stm}.

The two most important properties of transactional memory are.
\begin{enumerate}
\item No deadlocks, because no locks are used.
\item \emph{Liveness}. The system \emph{as a whole} always makes progress, even
  in the presence of conflicts.
\end{enumerate}

Transactional memory makes it much easier to write correct concurrent programs.
Moreover it does this with minimal adjustments to the original code.
Relating this back to the vector example from above, if an STM compiler is used,
the only change to the code for the append routine would be to enclose it in a
\texttt{BEGIN\_TRANSACTION}, \texttt{END\_TRANSACTION} call.

Despite its advantages, transactional memory is not very widely used, with one
notable exception: \emph{Haskell}. The package database \emph{Stackage} lists
the \texttt{stm} library as a dependency for 692~\cite{stm-as-dep-on-stackage}
open source Haskell packages\footnote{To put this into perspective, in the same
  package database the \emph{directory} library for platform agnostic filesystem
  operations, is listed as dependency for 1906
  packages~\cite{directory-as-dep-on-stackage}.}. Notable among those are the
\emph{AGDA} compiler, the build tools \emph{Cabal} and \emph{stack} and the
webframework \emph{snap}.

In this paper I take a look at STM as implemented in Haskell. I highlight how
this implementation uses the Haskell type system to solve some of the common
issues of STM. I also show how the uncommon paradigm of immutability in Haskell
is used to make its STM implementation more efficient.

\section{Haskell}

\label{sec:haskell}

Many of the examples throughout this paper use Haskell syntax. Most prominently
for type annotations. Following therefore is a short introduction to selected
aspects of the Haskell syntax.

Values are annotated with types using the \texttt{value :: Type} syntax,
regardless of whether it is a function or other value. Function types are
expressed with the \texttt{->} which separates both the arguments and the return
type. The type to the right of the last \texttt{->} is always the return type,
all other types are arguments. Thus \texttt{f :: Int -> String -> Bool} reads as
\texttt{bool f(int i, string s)} in C/C++. Perhaps confusing is the omission of
parameter names in the Haskell type signature. Applying functions to arguments
in Haskell is done in simple prefix notation, with no parentheses and no commas.
Thus \texttt{writeFile("filename", content)} would translate to
\texttt{writeFile "filename" content}.

In the Haskell data is immutable by default. Regular data structures cannot be
modified, instead new data is allocated. Similarly binding ``variables'' is
final. The \texttt{let x = 1} syntax declares a new variable \texttt{x}. Though
``variable'' may be the wrong word here, because there is no way to change this
value. Instead new data must be created, for instance \texttt{let y = x + 1}.

In addition side effects are strictly controlled. Functions are either pure or
must explicitly declare the side effects they perform. Pure functions have the
property that for the same inputs they always produce the same outputs. They
cannot modify any external state, perform I/O or read mutable variables. As an
example a function \texttt{randomInt :: () -> Int} producing a random integer
would be impossible in Haskell.

The return type of a function shows the side effects it performs. For instance
\texttt{readFile :: FilePath -> IO String} does I/O side effect by opening and
reading a file, and then returns a \texttt{String}. There are also functions
which only perform side effects. They return \texttt{()}, the Haskell equivalent
of \texttt{void}. An example would be \texttt{writeFile :: FilePath -> String ->
  IO ()}. The \texttt{IO} ``wrapper'' type around the string cannot be removed.
However Haskell provides the \texttt{do} notation to sequence many such
computations into larger ones. Inside the \texttt{do} block values produced by
I/O can be captured and assigned to names with \texttt{<-}. These values can be
manipulated as usual. However the \texttt{IO} type propagates to the return type
of the computation. As an example see Figure~\ref{fig:simple-do}.


\begin{figure}
  \begin{lstlisting}
cCompile :: IO ()
cCompile = do {
  header <- readFile "main.h";
  code <- readFile "main.c";
  let object = runCompiler header code;
  writeFile "main.o" object;
  return ()
}
  \end{lstlisting}
  \caption{Example for the \texttt{do} notation}
  \label{fig:simple-do}
\end{figure}


The only way in Haskell to perform the actual effects of an \texttt{IO}
computation is to use it in the \texttt{main} function. Because effect types,
such as \texttt{IO}, propagate automatically, we know that a pure function can
never perform I/O.

In Haskell raw memory manipulations are not supported. Instead there are special
data structures which represent mutable memory cells, called \texttt{IORef}.
Instead of primitive, syntactic constructs, there is a suite of functions to
manipulate them, shown in Figure~\ref{fig:io-ref}.

\begin{figure}
  \begin{lstlisting}
data IORef = {- implementation hidden -}
newIORef :: a -> IO (IORef a)
readIORef :: IORef a -> IO a
putIORef :: IORef a -> a -> IO ()
  \end{lstlisting}
  \caption{Mutable memory cell in Haskell}
  \label{fig:io-ref}
\end{figure}

These manipulations are also considered side effects, as can be seen from the
\texttt{IO} return type. As a result they also can only be used in an
\texttt{IO} context.

\section{The Haskell STM Library}

\label{sec:library}

Haskell exposes its STM implementation via a library called
\texttt{stm}~\cite{composable-transactions}. The underlying STM system is
implemented in the Haskell runtime and similar to the one described in
Section~\ref{sec:stm}. Though it has been shown that a it can also be
implemented in Haskell itself, using lower level concurrency
primitives~\cite{stm-in-concurrent-haskell,concurrent-haskell}.

The library exposes a high level interface to the
programmer. The advantage is that it can use the Haskell type system to enforce
correct use of transactional data and side effect freedom.
Sections~\ref{sec:side-effects}~and~\ref{sec:explicit-implicit} describe these
in more detail.

The library is designed around explicit transactional data structures. The
simplest of these is called \texttt{TVar}, a mutable memory cell that can be
read or written. The interface for interacting with a \texttt{TVar} can be seen
in Figure~\ref{fig:tvar}. The interface for \texttt{TVar} is deliberately
similar to the one for \texttt{IORef} from Section~\ref{sec:haskell}. The entire
library mimics preexisting, non-transactional Haskell data structures. The
transactional lock \texttt{TMVar} for instance is similar to an \texttt{MVar}, a
non-transactional lock. By making names and interfaces similar to
familiar data structures \texttt{stm} reduces the barrier to entry for Haskell
programmers already familiar with concurrent Haskell~\cite{concurrent-haskell}.

The second important part of the library is the transaction type \texttt{STM}.
\texttt{STM} is an effect type, similar to \texttt{IO}. It is used to build up
transactions from primitive operations using the \texttt{do} notation. It
combines both primitives, as well as other fragments of transactions together.
Thus libraries can define only partial transactions which the user can combine
later with their own logic using \texttt{do}.

\begin{figure}
  \begin{lstlisting}
atomically :: STM a -> IO a
atomically transaction = do {
  START_TRANSACTION;
  result <- runTransaction transaction;
  END_TRANSACTION;
  return result
}
  \end{lstlisting}
  \caption{A rough version of the \texttt{atomically} function}
  \label{fig:atomically}
\end{figure}

To execute the transaction the \texttt{atomically} function is provided, see
also Figure~\ref{fig:atomically}. It takes a description of a transaction
(\texttt{STM a}) and executes it. The \texttt{atomically} function both
initializes the transaction as well as verifies it.

\begin{figure}
  \begin{lstlisting}
data TVar = {- implementation hidden -}
newTVar :: a -> STM (TVar a)
readTVar :: TVar a -> STM a
putTVar :: TVar a -> a -> STM ()
  \end{lstlisting}
  \caption{STM memory cell in Haskell}
  \label{fig:tvar}
\end{figure}

The Haskell \texttt{stm} library exports three additional primitives.
\texttt{retry}, \texttt{orElse} and \texttt{check}. These were first to appear
in this
library~\cite{composable-transactions,transactional-memory-data-invariants} and
are not part of the standard STM interface. \texttt{retry} is a way for the user
to demand the transaction be aborted and attempted again. This function allows
the programmer to require more complex conditions to hold in the system. Take
for instance a function which must read two elements from a shared buffer. The
implementation in Figure~\ref{fig:dequeue-two} uses \texttt{retry} if any of the
buffers are empty. When a \texttt{retry} is called the STM implementation rolls
back the transaction and then waits for a change to occur on any of the
locations touched until \texttt{retry} is called. This avoids executing the
computation again when the conditions have not changed. The use of a
\texttt{TMVar} as a transactional lock here is safe, as it is tracked by the STM
runtime and automatically released should the transaction abort.

\begin{figure}
  \begin{lstlisting}
dequeue :: TMVar [a] -> STM a
dequeue v = do {
  l <- takeTMVar v; -- acquire lock
  if empty l
    then retry
    else do {
      putTMVar v (tail l); -- release lock
      return (head l)
    }
}

dequeueTwo :: TMVar [a] -> STM (a,a)
dequeueTwo v = do {
  a1 <- dequeue v;
  a2 <- dequeue v;
  return (a1, a2)
}
  \end{lstlisting}
  \caption{Dequeuing a buffer}
  \label{fig:dequeue-two}
\end{figure}

The second primitive added is the \texttt{orElse} function. \texttt{orElse}
takes two STM computation, tries the first and if it \texttt{retry}s it
executes the second. If we understand \texttt{retry} as a blocking transaction,
because it waits for change, then \texttt{orElse} is the choice of two blocking
transactions. By using \texttt{orElse} we can wait on two different conditions
for either one to come true. The authors of the original paper note the
similarity to the unix \texttt{select} system call.

Lastly the library introduces the \texttt{check} function which allows the
programmer to specify an ``expression that should be preserved by every atomic
update for the remainder of the program’s
execution''~\cite[abstract]{transactional-memory-data-invariants}. This allows the
user to programmatically state conditions which must be preserved in the
transaction. The STM system can then optimise waiting and execution of
transactions to suite these conditions.

To sum up, Haskell exposes its STM implementation via a library. The interface
is high level and mimics similar, non-transactional data structures. In addition
to the ``regular'' STM primitives it introduces thee new functions which add
blocking, choice and programmatic condition checking to transactions. Neither of
these three is part of the standard STM interface. They enable users to express
even more complex computations in a succinct way and delegate the responsibility
for efficiency to the STM system. Concrete examples can be found in the two
publications~\cite{composable-transactions,transactional-memory-data-invariants}.

\section{Manual and Automatic STM}

\label{sec:explicit-implicit}

The problem with manual STM, where the programmer inserts \texttt{READ} and
\texttt{WRITE} calls, is that these can be forgotten without raising an error.
Special STM compilers are employed to automate the insertion of \texttt{READ}s
and \texttt{WRITE}s this is often also necessary to instrument third party
libraries that are used inside transactions. Since the STM protected accesses
are much more expensive than regular memory accesses protecting data that is not
shared should be avoided. The process of data that was shared becoming local to
the thread is called \emph{privatization}. Identifying which is private is not
trivial and involves sophisticated escape analysis. As a result STM compilers
are generally too conservative and protect additional memory that is not shared.
This further reduces their performance.

Manual STM can be used to avoid this excess protection. However in such cases it
is easy to forget protecting actually shared memory, because most languages make
no type level distinction between protected and unprotected memory. In Haskell
STM protected memory structures are distinct types. The simplest is a mutable
memory cell called \texttt{TVar} (see Figure~\ref{fig:tvar}). Its internal
structure is hidden from the user of the library which prevents any raw
manipulation of a \texttt{TVar} value. It can only be manipulated via the
functions provided by the library. These \texttt{readTVar} and \texttt{putTVar}
functions internally use \texttt{READ} and \texttt{WRITE} to protect the memory.

No special STM compiler is needed here. The programmer uses STM reads and writes
explicitly and the restricted library interface ensures that. It also means safe
privatization. While the \texttt{TVar} is itself mutable the value stored there
is either immutable or another explicit \texttt{TVar}. Once the value of a
\texttt{TVar} has been read its members can be accessed without using the costly
STM protection. If one of the members happens to be mutable (aka a
\texttt{TVar}) only \emph{that} access is again protected by the system.

The type system further ensures \texttt{TVar}s are only used inside
transactions. Modification of shared, protected memory from outside a
transaction is a problem for stm implementations~\cite{research-toy}. Most
implementations simply do not protect against this. As shown in
Figure~\ref{fig:tvar} the functions for manipulating \texttt{TVar}s all have the
\texttt{STM} type which can only be used inside a transaction.

\texttt{TVar}s themselves can be returned from a transaction. For instance by
doing \texttt{atomically (newTVar 0) :: IO (TVar Int)}. This way \texttt{TVar}s
can be passed around outside of transactions, but only read or written to in
another transaction.

Having only manual STM in Haskell unfortunately means that STM cannot easily
applied to third party libraries. However this is rarely a problem in practice,
because many libraries in Haskell tend to be designed in terms of pure functions
which means they do not need to involve STM.

\section{Side Effects}

\label{sec:side-effects}

In order for an STM system to function properly transactions must be free of
side effects. Side effects are actions that manipulate external state. This
includes reading and writing files but also manipulations of global memory
locations. The problem is that the semantics of side effects in transactions are
unspecified and it is easy to use them unsafely. When a transaction is retried
its side effects are executed again. Furthermore there is no guarantee
\emph{which} of its side effects are executed again. The transaction may only
run partially and then abort, executing only a part of the side effects. They
also interleave with the side effects from concurrent transactions. As because
of this programmers are generally discouraged from using side effects in
transactions.

Detecting the use of side effects is difficult. It is possible to detect the use
of syscalls at runtime. Systems which do so will usually abort with an error to
inform the user. Much more complicated is the detection of memory modifications.
For efficiencies sake it is necessary to allow raw memory modifications in
transactions. However writes must be restricted to memory created in the
transaction. Furthermore this memory must not have been written out to a
protected location. The usual tactic of STM compilers is to use escape analysis,
which is not a trivial problem.

Haskell deals with this problem differently. The type system and the standard
library in Haskell make side effects explicit (see Section~\ref{sec:haskell}).
Memory manipulations (see Figure~\ref{fig:io-ref}) and OS interactions (see
Section~\ref{sec:haskell}) have the \texttt{IO} type. This type is incompatible
with the type of memory transactions \texttt{STM} (see Figure~\ref{fig:tvar}).
Both of these types are effect types. By default effect types cannot be
interleaved or embedded. An explicit function must be provided to embed one
effect in another. \texttt{STM} effect can be embedded into \texttt{IO} effects
using \texttt{atomically} (compare Figure~\ref{fig:atomically}). However there
is no such function for the other direction. \texttt{embedIO :: IO a -> STM a}
would be an example for such a function. However authors of the \texttt{stm}
library deliberately do not include such a function. Furthermore since both the
implementation of \texttt{IO} as well as \texttt{STM} are hidden, the user
cannot simply define such a function themselves.

As a result the compiler will detect and report the use of \texttt{IO}
functions, and therefore the use of side effects, in an stm transaction. This
prevents the accidental use of side effect performing functions. Haskell is in
the fortunate position to already have explicit side effects which makes this
approach possible int he first place.

\section{Related Work}

\label{sec:related-work}

While STM has not caught on much outside of Haskell, other concurrency
primitives have.

\subsection{HLE}

Hardware lock elision~\footnote{In the first occurrence of this idea it was
  called Speculative Lock Elision~\cite{sle}}, or HLE is a hardware mechanism
that speeds up lock based concurrent programs~\cite{sle,hle}. The idea is rather
similar to transactional memory. A lock based implementation is executed
optimistically without acquiring the lock. When the lock would be released a
check for conflicts is performed using a cache coherence protocol. ``Two memory
operations conflict if they concurrently access the same cache line and one of
them is a write.''~\cite{hle} If no conflict occurs the writes are persisted. If
a conflict occurs the threads re-execute the section non-speculatively. This
only works if there are no side effects such as IO in the lock protected
sections.

The problems are similar to TM, in that it needs hardware capabilities, and that
it is limited by the size of the cache. However unlike HTM, HLE is a runtime
feature of the processor. No changes to the code need to be performed, making it
suitable for use with legacy systems and third party libraries as well. This
also makes it, if not portable, compatible. On systems which do not support HLE
it automatically falls back to the less efficient, lock based solution.

\subsection{Message passing}

Message passing, and by extension the \emph{actor model} is most up-and-coming
concurrency model. This is often coupled with lightweight threads that are
scheduled by a runtime. Threads only interact with private state. Between them
there are channels through which data can be sent. While the paradigm itself is
quire old~\cite{mpi-paradigm,actors} it has only recently found its way into the
mainstream of programming. The first language to use this model extensively was
Erlang, a language specifically designed for very reliable, highly distributed
systems. It is not a widely used language, but neither irrelevant due to its use
in industry. Nowadays the standard library for Go and Rust both contain an
implementation of message passing. And it has also found its way onto the Java
platform, notably with the Akka framework~\cite{akka}.

The advantage of MP is that these principles are easy to understand and they
scale well. Some implementations, such as Erlang\cite{erlang}, require messages
to be serializable, an thus individual nodes can be moved to other processes or
even machines making them scalable beyond threads.

A common problem of MP is the serialization overhead of messages. This can be
avoided by sending pointers directly when communicating between threads. However
care must be taken that the access of the sending thread is revoked~\footnote{Go
  for instance allows you to send pointers through channels. This is efficient,
  but the recipient can modify the memory pointed to which, since it was sent
  may be shared. There is no mechanism in the compiler to detect and/or report
  this.}. Rust has a rather elegant solution here where its ownership type
system ensures that no two threads access the same data concurrently.

\subsection{Hybrid TM}

While not strictly a ``different'' concurrency primitive it is worth mentioning
that there are designs for so called hybrid transactional systems. These combine
HTM and STM. One way of doing this is by using HTM wherever possible and if the
the hardware capacity is exceeded fall back to a software based
solution~\cite{htm-with-stm-fallback}.

Alternatively hardware is used to speed up an STM
system~\cite{hardware-accelerated-stm}. Here a classical STM system is employed
but the instruction set of the machine is extended with new instructions
specifically designed to aid the STM implementation. The claim is that this way
near HTM performance can be achieved while maintaining the flexibility of STM.

None of these are seriously in use yet.
\section{Conclusion}

\label{sec:conclusion}

Languages such as Haskell are a better fit for the use of STM. Some of the
common problems with STM do not occur in Haskell. Immutable data and explicit
memory manipulation remove the need for complicated privatization and overly
conservative protection. Furthermore side effects in transactions are
categorically forbidden via the type system.

STM does not solve the concurrency problem completely. Even in Haskell there are
other primitives available, such as message passing and a lock-like
data structure. Both of which are also widely used. However STM is perhaps
currently the safest, easiest and most reliable system to express computations
with complex invariants.


\bibliographystyle{ACM-Reference-Format}
\bibliography{bibliography}

\end{document}
